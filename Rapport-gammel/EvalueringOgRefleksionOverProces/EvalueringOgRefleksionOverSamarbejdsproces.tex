\section{Evaluering og reflektion over proces}
\label{sec:evaluering_og_reflektion_over_proces}

I dette afsnit reflekteres og evalueres der over hele processen fra problemidentifikation til produkt. 

I den indledende fase udarbejdede gruppen i fællesskab en samarbejdsaftale, hvori der blev sat prioriteringer, ambitionsniveau, præferencer m.m. Gruppen besluttede, at der efter hvert møde skulle føres log og udfyldes arbejdsblade, samt planlægge arbejdsopgaverne for næste gang. Samarbejdsaftalen var med til at afklare hvilke forventninger, som vi hver især havde til samarbejdet, og var med til at sætte fælles mål for projektet.

Der kunne nu tages hul på den næste fase, hvor problemet identificeres og defineres. Her foretog gruppen en grundig gennemgang af de obligatoriske krav samt hvilke valgfrie krav, der kunne tilføjes. Dernæst afgrænsede gruppen sig i første omgang til at tilføje en zoom-ud funktion udover de obligatoriske krav. Det var på daværende tidspunkt svært at tage stilling til, hvilke valgfrie funktioner vi ville tilføje. Det vigtigste for os var at kunne få lavet et første udkast af visualisering af kortet. På baggrund af dette var gruppen afklaret med at benytte Krak's data som datakilde, da den var nemmere at benytte og håndtere.   

Det næste trin i forløbet var at opstille en løsningsmodel, hvor gruppen tog udgangspunkt i at strukturere koden i et Model-view-controller designmønster. Vi foretog en tavlediskussionen, der omhandlede hvilke klasser der skulle være i hvert modul. Dette var med til at give gruppen et godt overblik over, hvad programmet skulle indeholde, hvilke arbejdsopgaver der skulle udføres, og hvilke centrale klasser der skulle laves. 

Efter at have opstillet en løsningsmodel, kunne vi arbejde hen imod at få lavet et første udkast af programmet. Det var ikke alle der fik skrevet kode i første omgang. Det var primært dem, som havde styr på at kode, der tog styringen. Dette kunne ikke undgås, da der i starten var få og store arbejdsopgaver, og da det var vigtigt for gruppen at opnå resultater på kortest tid. Grundet de få arbejdsopgaver opstod der tilfælde, hvor kun én person skrev kode, mens de andre fulgte med og kom med input. Dette medførte, at produktiviteten var relativ lav. Det var også denne del, der var tungest og tog længst tid. 

Produktiviteten begyndte først at blive relativt høj efter, at vi fik tegnet kortet for første gang. Der blev arbejdet på forskellige moduler på én gang, og Model-view-controller designmønstret gav stor glæde og viste sig her at være særlig nyttigt. Der blev i gruppen arbejdet mere effektivt, og der blev tilføjet flere valgfrie funktioner. Vi fik i gruppen mere overskud og større overblik og kunne lave væsentlige ændringer med henblik på at nå til  et egentligt produkt. 

For at have undgået den lave produktivitet, kunne vi have lagt en bedre strategi. Vi kunne have dannet os et større overblik og overvejet om det var muligt at nedbryde nogle klasser til flere klasser samt evt. lavet interfaces. Vi har i gruppen ikke haft den store erfaring med at lave større programmer, hvilket bl.a. er en af grundene til, at vi stødte ind i denne problematik.