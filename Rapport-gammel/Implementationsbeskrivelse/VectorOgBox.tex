\section{Vector}

Klassen \emph{Vector} er en datastruktur, der repræsenterer en vektor. Klassen indeholder en række hjælpemetoder, der bl.a. tillader beregninger med vektorer.

\begin{itemize}
	\item \textbf{dist(Vector v)}: Beregner afstanden mellem to vektorer.
	\item \textbf{abs()}: Konverterer en vektors værdier til de tilsvarende absolutte værdier.
	\item \textbf{translate(Box from, Box to)}: Tager to bokse og skalerer vektoren fra den ene boks, som den befinder sig i, til den anden boks.
	\item \textbf{mirrorY(Box box)}: Spejler vektoren i forhold til y-aksen i boksen.
	\item \textbf{isInside(Box box)}: Tager en boks som parameter, og tjekker om vektoren befinder sig i boksen.
	\item \textbf{toArray()}: Returnerer værdien af x og y for en vektor i en tabel (array).
\end{itemize}

Klassen gør det også muligt at udføre aritmetiske operationer som addition, subtraktion, multiplikation og division. 

\section{Box}

Klassen \emph{Box} er ligeledes en datastruktur. En box består af to vektorer, start og stop, der udgør henholdsvis øverste venstre hjørne og nedeste højre hjørne af boxen.

Derudover indeholder klassen en række metoder:

\begin{itemize}
	\item \textbf{dimensions()}: Returnerer boksens dimensioner.
	\item \textbf{translate(Box from, Box to)}: Skalerer fra from-boksen til to-boxen.
	\item \textbf{flipX()}: Bytter om på x-koordinaterne i start og stop.
	\item \textbf{flipY()}:  Bytter o på y-koordinaterne i start og stop.
	\item \textbf{properCorners()}: Benytter flip-funktionerne til at sikre at start og stop repræsenterer de rigtige vektorer.
	\item \textbf{getCenter()}: Finder centrum af en boks.
	\item \textbf{scale(double scalar)}: Skalerer boksen i forhold til den givne skalar.
	\item \textbf{relativeToAbsolute(Vector relative)}: Konverterer vektorens relative koordinater, i forhold til boksen, til absolutte koordinater.
	\item \textbf{absoluteToRelative(Vector absolute)}:Konverterer vektorens absolutte koordinater til relative koordinater, i forhold til boksen.
	\item \textbf{overlapping(Box box)}: Afgør om boksen overlappes af en anden boks.
	\item \textbf{ratio()}: Returnerer den relative forskel mellem højden og bredden af boxen som en vektor.
	\item \textbf{toArray()}: Returnerer boksens koordinater i et to-dimensionelt array.
	\item \textbf{grow(double amount)}: Udvider boxen i både længden og bredden.
\end{itemize}