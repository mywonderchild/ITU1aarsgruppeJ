\section{Datastruktur}
\label{sec:datastruktur}
I implementeringen af ethvert kort må den bagvedliggende datastruktur nøje overvejes, da denne i høj grad afgør applikationens ressourcekrav. Overvejelserne omkring datastrukturen involverede blandt andet muligheden for kun at indlæse specifikke dele af dataet samt søgeegenskaber. Det givne datasæt består af knudepunkter og tilhørende vejsegmenter, som forbinder to af disse knudepunkter. Knudepunkterne indeholder information om deres individuelle placering, mens vejsegmenterne er forbundet med en række forskellige oplysninger såsom vejnavn og -type. Da kortdataet består af vejsegmenter, er det også af betydning for valget af datastruktur. Værd at overveje var også datastørrelsen --- datakilden fra Krak indeholder over en million knudepunkter, og disse skal på ethvert tidspunkt under programkørslen kunne tilgås, hurtigt.

To forskellige datastrukturer syntes at besidde de nævnte egenskaber: en sorteret tabel og en quadtree-struktur. Den sorterede tabel brillerer med en ukompliceret implementering. En quadtree-struktur, mens kompliceret at implementere, udmærker sig derimod ved dens simple datasegmentering samt hastig indlæsning af specifikke områder. 

Det vurderedes, at quadtree-strukturen ville udmunde i en både hurtigere og mere dynamisk datastruktur og således sås som den mest passende løsning.
% Forklar mere specifikt, hvad et quadtree er og hvordan det virker?