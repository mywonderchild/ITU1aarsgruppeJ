\section{Nærmeste vej}
For at implementere nærmeste vej benyttede vi en søgealgoritme i datastrukturen. Udformningen af søgealgoritmen lå imidlertid ikke fast, da flere forskellige løsningsforslag blev fremsat. Datamodellen var implementeret med en quadtree-struktur (se \ref{sec:datastruktur}), hvilket havde afgørende betydning for søgealgoritmens udformning. Vores første indskydelse var at gennemløbe samtlige synlige vejsegmenter, og finde ud af hvilken, der var nærmest musemarkøren. Det var dog indlysende, at denne løsning ville være langsom. Derfor fandt vi i stedet en løsning, som byggede på allerede eksisterende funktionalitet i datastrukturen --- nemlig rektangulære forespørgsler på data. Et løsningsforslag blev således, at forespørge om et indledende rektangel med en fastsat størrelse omkring brugerens markør for derefter at udvide dennes dimensioner indtil en eller flere veje findes. Sammenlignet med det førnævnte løsningsforslag med nabosøgning vurderedes denne løsning at være langsommere, men da implementeringen af nabosøgning blev betragtet som værende meget besværlig, valgte vi at arbejde videre med den langsommere løsning af arbejdstidshensyn. Det vurderedes endvidere, at hastighedsforskellen mellem løsningerne ville vise sig at være ubetydelig, hvilket styrkede vores valg af søgealgoritme.