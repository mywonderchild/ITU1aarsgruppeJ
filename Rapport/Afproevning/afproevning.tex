\section{Afprøvning}
\label{sec:afproevning}
I bestræbelse efter sikring af applikationens kvalitet og pålidelighed er en række afprøvninger af dennes bestanddele udført løbende med udviklingen. Afprøvningen er ikke foretaget med en målsætning om at udtømme samtlige muligheder for systemfejl, men derimod blot for at forhindre defekter, i de afprøvede klasser. Derudover fandt vi det ofte meget nyttigt at teste vores kode løbende under udviklingsprocessen, da dette ofte gjorde udviklingen af nye funktionaliteter mere effektiv.

\subsection{Metode}
\label{subsec:metode}
% evt. nævn, at der findes et hav af testmetoder
Da afprøvningsmetoden, som anvendes under udarbejdelsen af tests, er afgørende for afprøvningens dækningsgrad, fastsattes den indledningsvist som \emph{black-box testing} med supplerende \emph{white-box testing} af udvalgte dele af kodebasen, hvor vi fandt det gavnligt.
% Det er således påkrævet, at samtlige offentlige metoder i en afprøvet klasse køres én gang som minimum.

Eftersom udviklingen er foretaget i Java, er afprøvning udført i prøvemiljøet JUnit --- et populært Java værktøj til netop afprøvning. JUnit er således anvendt under kørsel af prøverne, hvorfor disse også er skrevet op imod JUnits API.

Som nævnt er formålet med afprøvningen ikke at gennemteste samtlige kodefragmenter i applikationen, hvorfor kun en række konkrete klasser og deres metoder er afprøvet. \emph{Vector} og \emph{Box} klasserne er brugt på tværs af applikationen, og netop derfor er disse udvalgt.

\subsection{Ækvivalensklasser}
\label{subsec:aekvivalensklasser}
Vector og Box klasserne er dataklasser, som indeholder værktøjsmetoder til hyppige operationer. Derfor håndterer samtlige metoder i bund og grund kommatal. Ækvivalensklasserne for afprøvning af klassernes metoder er således negative input, positive input samt nul-input.

\subsection{Dokumentation}
\label{subsec:dokumentation}
Dokumentation for de udførte prøver findes i bilag \ref{sec:forventningstabeller}. Her ses de forventede og faktiske resultater af afprøvningsmetoderne opstillet i forventningstabeller.