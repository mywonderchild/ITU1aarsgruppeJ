I dette afsnit fordybes der i de designvalg der er blevet foretaget til brugergrænsefladen, samt hvilke overvejelser der lå bag de valg.

\section{Visualisering}

\subsection{Vejfarver og -tykkelser}

Når en bruger benytter sig af kortet, er det en forudsætning, at brugeren skal kunne skelne mellem forskellige vejtyper. Særligt relevant bliver dette, når ruteplanlægningsfunktionen benyttes, da brugeren skal kunne skelne mellem motorveje og hovedveje. Den oplagte løsning er at adskille vejtyperne fra hinanden ved at tildele vejene forskellige farver, hvilket også er et af kravene stillet i kravsspecifikationen. 

Et kort, hvor vejene differentieres efter farver, siger ikke nok om deres vejtyper. Det vil kræve at brugeren på forhånd kender til hvilke farver, der er knyttet til vejtyperne. 

En større adskillelse af vejtyperne kan opnås ved at tegne dem med forskellige vejtykkelser. Med dette design skabes et udtryk der reflekterer virkeligheden, hvilket gør det lettere for brugeren at skelne mellem vejene. 


\subsection{Panorering}

Panorering er en central funktion i et kortprogram. Vi overvejede 2 forskellige måder at implementere panorering på. Dette kunne enten være at trække i kortet eller trykke på visuelle piletaster. En beslutning, om at der ikke skal være et visuel interaktionssystem, i form af taster og zoom knapper, blev truffet tidligt i processen. Et visuelt interaktionssystem kunne blænde brugeren for at benytte andre måder at interagere med programmet på. Derudover benytter en bruger sig typisk af en mus. Det kunne fremstå som værende forstyrrende, hvis der er piletaster på skærmen.  

På baggrund af ovenstående overvejelser valgte vi, at panorering skal foregå ved at trække i kortet. Tastaturets piletaster kan også benyttes, der udgør et alternativ, som er en naturlig måde at navigere i kortet, som ikke er forstyrrende. 


\subsection{Zoom}
En af de helt store funktioner er zoom-funktionen. Der skulle her tages højde for, hvordan denne skulle komme til udtryk hos brugeren samtidig med at gøre det bekvemt at benytte.  

Der kan zoomes på følgende måder:

\begin{itemize}
	\item Zoom med taster.
	\item Zoom med mus
	\item Rektangulær zoom.
\end{itemize}

Når der zoomes med taster eller mus, så sker det relativt til centrum af det aktuelle udsnit. Den anden mulig løsning havde været, hvor der zoomes relativt til musemarkør. Da vi ikke kunne implementere denne løsning. Von Malthe synes også det er en irriterende zoomfunktion, så det er ok.

Hvis brugeren hurtigt vil zoome ind på et specifikt udsnit af kortet, så er de to første zoom metoder ikke tilstrækkelige. Med rektangulær zoom kommer man hurtigere til det ønskede udsnit. Det er en særdeles gunstig løsning til hurtig og målrettet zoom. 

\subsubsection{Synlige vejtyper}

Ved et zoomniveau, hvor eksempelvis hele Danmark er synligt, er det kun relevant at få fremvist motorveje og hovedveje. Man kan med fordel undlade at synliggøre mindre og ubetydelige veje i sådan et zoomniveau. Brugeren er ikke interesseret i at se de mindre veje, og det skaber et mere overskueligt billede at undlade dem.

Ved større zoomniveau synliggøres de mindre veje, da man må antage, at det er hvad brugeren ønsker at få fremvist.   

\subsubsection{Dynamisk vejtykkelse ved zoom}

For yderligere at give programmet et overskueligt udtryk, har vi tilføjet en mindre visuel funktion. Ved zoom vokser vejtykkelsen dynamisk. Det giver zoomfunktionen en lækker detalje og programmet et pænt look. 
  

\subsection{Visning af nærmeste vejnavn}

Musemarkøren vil altid vise det nærmeste vejnavn.

Vi havde to løsningsforslag til at vise nærmeste vejnavn. 

\begin{enumerate}
   \item Vise vejnavn ved musemarkøren.
   \item Vise vejnavn i en statusbar.
\end{enumerate}

Den første løsning kan forstyrre brugeren, da vejnavnene konstant vil ligge over kortet, nær musemarkøren, og derved være i vejen. I statusbaren er det ikke lige så iøjenfaldende. 
	
\subsection{OpenStreetMap eller Krak}
Et af de obligatoriske krav til programmet er, at brugeren kan benytte sig af enten Krak- eller OpenStreetMap-datasættet. Det oplagte er, at brugeren får denne valgmulighed ved opstart af programmet. I tilfælde af at brugeren ønsker at skifte datasæt i et kørende program, kunne man give brugeren for dette. Vi valgte dog at nedprioritere denne mulighed. For at skifte mellem datasæt må brugeren lukke programmet ned, genåbne det og derfra vælge det datasæt, brugeren ønsker at benytte. 


\section{Rutevejledning}

\subsubsection{Rute planlægning}

Brugeren kan få planlagt en rute på følgende to måder:

\begin{itemize}
	\item Klikke to steder på kortet.
	\item  Indtastning af udgangspunkt og destination i sidebaren.
\end{itemize}

At klikke to steder på kortet er en indlysende måde at få lavet en rutevejledning. Præcisionen af denne form for søgning kan variere, alt afhængig hvilket zoomniveau brugeren befinder sig på. Hvis der er zoomet betydeligt ud, falder præcisionen markant. Når der er zoomet tilpas ind, er dette en udemærket måde at planlægge sin rute på. En udvidelse til denne metode, ville være at markere klikkene med start og slut punkter. For en mere nøjagtig ruteplanlægning kan man benytte sidebaren.

Uanset hvilken måde der benyttes, kunne implementering af husnumre øge præcisionen til ruteplanlægning. 

\subsection{Sidebar}

I sidebaren indtastes udgangspunkt og destination, ligeledes udskrives rutevejledningen i denne. 

Sidebaren fylder meget, og hvis den er fremvist konstant, kan det ødelægge oplevelsen af programmet. Det er ikke bekvemt at have sidebaren fremme, når man navigere i kortet. 

Det skal derfor være muligt at skjule sidebaren, når den ikke bruges, og få den fremvist, når den skal benyttes. Til dette er der en diskret knap nede i venstre hjørne.

\subsubsection{Adressegenkendelse}

Programmet vil ikke kunne beregne en rute, hvis der er blevet indtastet et vejnavn, der ikke er defineret i datasættet. Derfor har vi implementeret, at når der indtastes data i tekstfeltet, vil der automatisk komme en rullemenu til syne. Menuen indeholder mulige adresser, som minder om brugerens indtastning. Der vil i menuen altid være fem muligheder præsenteret i prioriteret rækkefølge, da det vil virke ubekvemt for brugeren at få præsenteret for mange muligheder.

Dette er med til at give brugeren hurtigere søgning og samtidig fjerne brugerens bekymringer om fejlindtastninger.

