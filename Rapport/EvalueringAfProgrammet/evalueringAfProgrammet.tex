Programmet understøtter alle de påkrævede features, såvel som flere af de valgfrie. Vi er derfor grundlæggende tilfredse med programmets funktionalitet. Vi kunne godt have tænkt os at have været mere kreative, og fundet på nogle features selv, men vi havde ikke det fornødne overskud.

\section{Hastighed}

Programmet er \~5 eller færre sekunder om at indlæse data ved opstart, hvilket vi synes er acceptabelt taget i betragtning af at data er lagret i rå tekstfiler.

Vi har arbejdet en hel del med at få applikationen til at have en hurtig opdateringsfrekvens uanset udsnit og type af brugerinteraktion, og det er vores indtryk at det langt hen ad vejen er lykkedes. Dog kan der opstå relativt lave framerates når der navigeres i særligt krævende udsnit af kortet (f.eks. helt zoomet ud, eller overblik over København).

\section{Fejl og mangler}

Vi har ikke umiddelbart noget kendskab til fejl og mangler.
% Kan det virkelig være rigtigt? Vi må da kende til en eller anden form for fejl eller mangel?

\section{Brugervenlighed}

Vores program giver ikke brugeren nogen informationer om hvad mulighederne for interaktion er, og det er derfor op til brugeren selv at udforske programmets features. Særligt galt kunne det gå hvis brugeren ikke er bekendt med digitale kort generelt, og derfor ikke har en forhåndviden om typiske muligheder for interaktion i denne type program.