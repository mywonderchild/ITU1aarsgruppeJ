\section{Evaluering og reflektion over proces}
\label{sec:evaluering_og_reflektion_over_proces}

I dette afsnit reflekteres og evalueres der over hele processen fra problemidentifikation til det midlertidige produkt. 

I den indledende fase udarbejdede gruppen i fællesskab en samarbejdsaftale, hvor der blev sat prioriteringer, et ambitionsniveau, præferencer m.m. Efter hvert møde skulle der føres log og udfylde arbejdsblade, samt planlægge arbejdsopgaverne for næste gang. Samarbejdsaftalen var med til at afklare hvilke forventninger man har til samarbejdet, og var med til at sætte et fælles mål for projektet.


Der kunne nu tages hul på den næste fase, hvor problemet identificeres og defineres. Her foretog gruppen en grundig gennemgang af de obligatoriske krav, samt hvilke valgfrie krav der kunne tilføjes.  Dernæst afgrænsede gruppen sig i første omgang til at tilføje en zoom-ud funktion udover de obligatoriske krav. Det var på daværende tidspunkt svært at tage stilling til, hvilke valgfrie funktioner vi ville tilføje. Det vigtigste for os var at kunne få lavet et første udkast af visualisering af kortet, der får fremvist vejene. På baggrund af dette var gruppen afklaret med at benytte KrakData som datakilde, da den var nemmere at benytte og håndtere.   


Det næste trin i forløbet var at få opstillet en løsningsmodel, hvor gruppen tog udgangspunkt i at strukturer koden i et Model View Controller framework. Vi foretog en tavlediskussionen der omhandlede hvilke klasser der skulle være i hver task. Dette var med til at give gruppen et godt overblik over hvad programmet skulle indeholde, hvilke arbejdsopgaver der skulle udføres og hvilke centrale klasser der skulle laves. 


Efter at have opstillet en løsningsmodel, kunne vi arbejde os hen imod at få lavet et første udkast af programmet. Det var ikke alle der fik skrevet kode i første omgang, det var primært dem, som havde styr på at kode, der tog styringen. Dette kunne ikke undgås, da der i starten var få og store arbejdsopgaver, og det var vigtigt for gruppen at opnå resultater på kortest tid. Grundet de få arbejdsopgaver opstod der tilfælde, hvor kun én person fik skrevet kode, mens de andre ville følge med og komme med input. Dette medførte at produktiviteten var lav, og når der opstod fejl i koden gik hele udviklingsprocessen i stå. Det var også denne del der var tungest og tog længst tid. 


Produktiviteten begyndte først at blive relativt høj efter at vi fik tegnet kortet for første gang. Der blev arbejdet på forskellige moduler på en gang og Model View Controller frameworket gav stor glæde og viste sig her at være særlig nyttigt. Der blev i gruppen arbejdet mere effektivt, og der blev tilføjet flere valgfrie funktioner. Vi fik i gruppen mere overskud og større overblik og kunne lave væsentlige ændringer med henblik på at nå det midlertidige produkt. 

For at undgå den lave produktivitet, kunne vi have lagt en bedre strategi. Vi kunne have fået dannet os et større overblik og overvejet om vi kunne nedbryde nogle klasser til flere klasser og evt. lave et interface. Vi har i gruppen ikke haft den store erfaring med at lave større programmer, og det bl.a. en af grundene til vi støtte ind på denne problematik. Gruppearbejdet har ellers fungeret godt og samarbejdsaftalen er tildels blevet overholdt. Der har været god kommunikation i gruppen og arbejdsmiljøet har været fornuftigt. 