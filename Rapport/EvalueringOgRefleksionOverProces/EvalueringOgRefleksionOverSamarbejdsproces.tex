\section{Evaluering og reflektion over proces}
\label{sec:evaluering_og_reflektion_over_proces}

I dette afsnit reflekteres og evalueres der over hele processen fra problemidentifikation til det endelige produkt. 

I den indledende fase udarbejdede gruppen i fællesskab en samarbejdsaftale, hvori der blev sat prioriteringer, ambitionsniveau, præferencer m.m. Gruppen besluttede, at der efter hvert møde skulle føres log og udfyldes arbejdsblade, samt planlægge arbejdsopgaverne for næste gang. Samarbejdsaftalen var med til at afklare hvilke forventninger, som vi hver især havde til samarbejdet, og var med til at sætte fælles mål for projektet.

Der kunne nu tages hul på den næste fase, hvor problemet identificeres og defineres. Her foretog gruppen en grundig gennemgang af de obligatoriske krav samt hvilke valgfrie krav, der kunne tilføjes. Dernæst afgrænsede gruppen sig i første omgang til at tilføje en zoom-ud funktion udover de obligatoriske krav. Det var på daværende tidspunkt svært at tage stilling til, hvilke valgfrie funktioner vi ville tilføje. Det vigtigste for os var at kunne få lavet et første udkast af visualisering af kortet. På baggrund af dette var gruppen afklaret med at benytte Krak's data som datakilde, da den var nemmere at benytte og håndtere.   

Det næste trin i forløbet var at opstille en løsningsmodel, hvor gruppen tog udgangspunkt i at strukturere koden i et Model-view-controller designmønster. Vi foretog en tavlediskussionen, der omhandlede hvilke klasser der skulle være i hvert modul. Dette var med til at give gruppen et godt overblik over, hvad programmet skulle indeholde, hvilke arbejdsopgaver der skulle udføres, og hvilke centrale klasser der skulle laves. 

Efter at have opstillet en løsningsmodel, kunne vi arbejde hen imod at få lavet et første udkast af programmet. Det var ikke alle der fik skrevet kode i første omgang. Det var primært dem, som havde styr på at kode, der tog styringen. Dette kunne ikke undgås, da der i starten var få og store arbejdsopgaver, og da det var vigtigt for gruppen at opnå resultater på kortest tid. Grundet de få arbejdsopgaver opstod der tilfælde, hvor kun én person skrev kode, mens de andre fulgte med og kom med input. Dette medførte, at produktiviteten var relativ lav. Det var også denne del, der var tungest og tog længst tid. 

Produktiviteten begyndte først at blive relativt høj efter, at vi fik tegnet kortet for første gang. Der blev arbejdet på forskellige moduler på én gang, og Model-view-controller designmønstret gav stor glæde og viste sig her at være særlig nyttigt. Der blev i gruppen arbejdet mere effektivt, og der blev tilføjet flere valgfrie funktioner. Vi fik i gruppen mere overskud og større overblik og kunne lave væsentlige ændringer med henblik på at nå til et egentligt produkt og færdiggøre første del af projektet. 

Kort efter aflevering af første del af projektet, opstod der en mindre krise i gruppen. Et gruppemedlem havde haft nogle personlige problemer, der var opstået under projektforløbet. De andre gruppemedlemmer kendte ikke til gruppemedlemmets situation. De kunne fornemme at ambitionsniveauet hos ham ikke var i overensstemmelse med det ambitionsniveau, som blev sat under samarbejdsaftalen. De andre gruppemedlemmer konfontrerede gruppemedlemmet. Gruppen fik renset luften og talt ud og gruppemedlemmet forklarede kort, hvad han havde gået igennem og samtidigt gjorde det klart for gruppen, at problemerne var blevet løst. Gruppen viste stor forståelse for gruppemedlemmets situation. Gruppen var igen på ligefod og var klædt på til anden del af projektet. 

Vi tog hul på anden del af projektet ved at gentage problemidentifikationsfasen. Der skulle tages højde for de nye obligatoriske krav og extentions som vores produkt skulle opfylde og indeholde. Det krævede at vi endnu en gang foretog en grundig gennemgang af de obligatoriske krav og extensions. I første omgang var målet blot at opfylde alle obligatoriske krav samt implementere én extension. Det var endnu uvidst hvilken extension vi ville implementerer, vi fik dog udelukket nogle extension, som vi ikke fandt interessante. 

Der blev i gruppen udarbejdet en arbejdsliste over de diverse arbejdsopgaver som skulle udføres. På daværende tidspunkt virkede det som en stor mundfuld med alt den funktionalitet, som yderligere skulle tilføjes til programmet. Størstedelen af alle arbejdsopgaver kunne løses uafhængigt af hinanden, dette gjorde det muligt at alle i gruppen kunne sidde med arbejdsopgaver. Dette var med til at øge produktiviteten, og gøre arbejdslisten mere overskuelig.  

Arbejdsopgaverne blev jævnt fordelt, det kunne ikke undgås at nogen fik lidt større og mere omfattende arbejdsopgaver end andre. Der var dog stor valgfrihed under fordeling af arbejdsopgaver. Efter fordelingen kunne vi påbegynde kodningen. Den overordnede arbejdsproces bestod nu af flere arbejdsforløb, som hver gruppemedlem var ansvarlig for. Der blev sat en endelig deadline til hvornår alle arbejdsopgaver skulle være færdige. Der var nu op til den enkelte gruppemedlem at overholde deadline. 

Denne arbejdsmåde var særdeles produktiv og havde sine fordele, men det havde også sine ulemper. Den store fordel var at når en gruppemedlem var færdig med sine arbejdsopgaver, kunne han hjælpe til med andre arbejdsopgaver eller fixe bugs og løse nyopståede problematikker. Et af de store ulemper var, at når man stødte på problemer, var det svært for de andre gruppemedlemmer at hjælpe, da de ikke var lige så godt sat ind i arbejdsopgaven som den ansvarlige. Dette medførte at der var perioder, hvor de enkelte arbejdsforløb var gået lidt i stå, og dette sænkede hele arbejdsprocessen. 

Da vi nåede den fastsatte deadline, havde vi færdiggjort stortset alle arbejdsopgaver. Vi manglede lidt på de ikke-gennemførte arbejdsopgaver, vi måtte dog erkende, at vi ikke helt var der hvor vi gerne ville være. Dette resulterede at vi måtte bruge mere tid på at få færdiggjort de sidste arbejdsopgaver og påbegyndelsen af rapportskrivning blev udskudt en smule. Der var ikke tid til at udføre de sidste forbedringer, som vi gerne ville have haft med. 

Efter den første del af projektet talte vi om, at lægge en bedre strategi for at kunne undgå den lave produktivitet. Vi kunne have dannet os et større overblik og overvejet om det var muligt at nedbryde nogle klasser til flere klasser osv. Derudover talte vi om, at vi i gruppen ikke haft den store erfaring med at lave større programmer, hvilket bl.a. er en af grundene til  vi stødte ind i denne problematik. 
Og det har vi taget med os.......
hvordan har det været?
hvad kunne vi gøre bedre?
kunne vi have undgået noget?
opsummering?