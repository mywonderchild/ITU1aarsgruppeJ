Denne rapport indeholder gruppe Js analyse og beskrivelse af den implementerede løsning af \emph{BFST Project}. Implementationen og rapporten blev udarbejdet i perioden fra d. 24/2-2014 til d. 21/5-2014.
Vi vil gerne takke Christian Nøhr Rasmussen, for at vi måtte benytte os af hans skalerede kystlinjedata, som vi anvender, når vi viser Krak-datasættet. Vi vil gerne takke Magnus Jacobsen og Sune Debel for god vejledning.

\section{Omkring Bilag}
I rapporten har vi vedlagt følgende bilag:
\begin{itemize}
	\item \textbf{Kravsspecifikation:} Den af kursusansvarlige udleverede PDF.\ref{sec:Kravsspecifikation}
	\item \textbf{Krakdokumentation:} Den medfølgende dokumentation til vores datapakke.\ref{sec:Krakdokumentation}
	\item \textbf{OSM-dokumentation:} Link til den officielle wiki for Open Street Map-formatet.\ref{sec:OSM-dokumentation}
	\item \textbf{Arbejdsfordeling:} Beskrivelse af gruppens arbejds- og ansvarsfordeling.\ref{sec:arbejdsfordeling}
	\item \textbf{Arbejdslog:} Vores log over den ugentlige proces.\ref{sec:Arbejdslog}
	\item \textbf{Arbejdsblade:} Vores log over det udførte arbejde.\ref{sec:Arbejdsblade}
	\item \textbf{Samarbejdsaftale:} Vores aftale omkring hvordan vi ville håndtere vores arbejdsforløb.\ref{sec:Samarbejdsaftale}
	\item \textbf{Forventningstabeller:} Oversigt over resultater fra programafprøvningen.\ref{sec:forventningstabeller}
\end{itemize}

