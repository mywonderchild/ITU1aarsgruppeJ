\section{Andre klasser}

Hermed en overfladisk gennemgang af de klasser, som ikke gennemgås i et af de dedikerede afsnit. Klasserne er delt op i moduler, og dernæst listet i alfabetisk orden.

\subsection{Generelle datastrukturer og hjælpeklasser}

\begin{itemize}
	\item \textbf{DataPurger:} Filtrerer overflødige kolonner fra den rå Krak-data og opretter nye datafiler.
	\item \textbf{Line:} Datastruktur der definerer en linje der er klar til at blive tegnet af \emph{Painter}.
	\item \textbf{Main:} Kører programmet.
\end{itemize}

\subsection{Model}

\begin{itemize}
	\item \textbf{Edge:} Datastruktur der definerer et vejsegment. Har også et par hjælpemetoder, der gør det nemt at initialisere \emph{Vector} objekter ud fra vejsegmentets data.
	\item \textbf{Groups:} Indeholder definitionen af hvilke typer, der hører til hvilke grupper, og kan afgøre hvilken gruppe et vejsegment tilhører (\emph{getGroup()}), samt hvilken farve en given gruppe har (\emph{getGroupColor()}).
	\item \textbf{Loader:} Indlæser data, opretter \emph{Nodes} og \emph{Edges} og arrangerer \emph{Edges} i \emph{QuadTrees}.
	\item \textbf{Node:} Datastruktur der definerer en knude på kortet.
	\item \textbf{QuadTree:} Datastruktur der indlæser \emph{Edges} rekursivt i en træstruktur bestående af kvadranter. Har også et par hjælpemetoder, der gør det nemt at hente \emph{Edges} ud af træet (\emph{queryRange()}) og bestemme hvilken \emph{Edge} i træet, der er tættest på et givent punkt (\emph{findClosest()}).
\end{itemize}

\subsection{Controller}

\begin{itemize}
	\item \textbf{KeyboardHandler:} Håndterer interaktion via tastetur.
	\item \textbf{ResizeHandler:} Opdaterer kortet når brugeren ændrer størrelsen på programmets vindue.
\end{itemize}

\subsection{View}

\begin{itemize}
	\item \textbf{Canvas:} Panel hvorpå kortet bliver tegnet. Holder styr på hvor lang tid det er siden, at kortet sidst blev tegnet.
	\item \textbf{Painter:} Tegner kortet på panelet. Dette gør den ved at modtage \emph{Graphics2D} objekter og modificere dem.
	\item \textbf{Window:} Programmets vindue der indeholder et \emph{Canvas}, hvor kortet bliver vist og en \emph{Label}, der viser hvilken vej, der er tættest på markøren.
\end{itemize}