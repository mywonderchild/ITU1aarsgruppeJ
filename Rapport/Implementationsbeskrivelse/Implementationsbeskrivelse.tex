\section{Model}
\label{sec:model}

Loader er ansvarlig for at indlæse data for det valgte datasæt og oprette instanser af Node og Edge objekter der anvendes som datastrukturer for henholdsvis punkter og vejstykker. Ligeledes er loader ansvarlig for at indsætte Edges i QuadTrees som gør det hurtigt at finde de Nodes og Edges der ledes efter.

Loader opretter flere forskellige QuadTrees der indeholder grupperinger af Edges. Groups er ansvarlige for at holde styr på hvilke typer af Edges der hører til hvilken gruppe, og hvilke egenskaber hver enkelt gruppe har. Derudover anvendes Node og Edge objekterne også til generering af både et adressekartotek og en graf til navigation. Grafen anvendes senere af AStar-klassen, som finder den hurtigste rute mellem to Nodes, mens adressekartoteket benyttes til tekstuel angivelse af start- og slutpunktet for rutenavigation.

\section{View}
\label{sec:view}

FirstWindow er et vindue der tilbyder brugeren valget mellem at bruge Krak eller OSM datasættet, er ansvarlig for at resten af programmet startes op med det valgte datasæt.

Window er vinduet som indeholder programmet efter at datasættet er indlæst. Canvas er det primære komponent i Window hvori kortet bliver tegnet.

Painter er ansvarlig for at tegne linjer og bokse og bliver anvendt af Tiler (se \ref{sec:controller}) og Canvas.

DropTextField tillader brugeren at vælge mellem en række forskellige adresseforslag gennem dennes rullemenu.

\section{Controller}
\label{sec:controller}

AdressButtonListener og AddressFieldListener tilsammen ansvarlige for at reagere på brugerinput i sidebaren i forbindelse med navigation. Når AddressFieldListener modtager information om at brugeren interagerer med de to tekst-input-felter kalder den AddressFinder for at hente forslag til adresser.

Når der er fundet en rute anvendes klassen Path som datastruktur og til at generere en tekstuel repræsentation af rutevejledningen.

KeyBoardHandler og MouseHandler er ansvarlige for at holde øje med brugerinput relateret til selve kortet. Disse to klasser står blandt andet for at håndtere når udsnittet skal zoomes, flyttes, eller nulstilles.

Tiler er ansvarlig for at tegne kortet, og holde styr på hvilket udsnit af kortet der vises.

\section{Tests}
\label{sec:tests}

Klasserne i denne pakke er tilsammen ansvarlige for at teste udvalgte klasser med JUnit tests.

\section{dataklargøring}
\subsection{OSM}
Vores saxParser  består af følgende klasser: Saxpurger, som er hovedfilen, man skal køre, saxEventhandler som håndterer langt de fleste dele af databehandlingen, saxFilter som filtrerer noget data fra og GeoConvert, der bidrager med hjælpefunktioner mod kortforvridning. SaxParseren løber igennem forskellige stadier, Først benytter den saxFilter, der frasorterer alle mærkater pånær nogle få specificerede, der indeholder den data vil skal bruge. Herefter henter den dataen ind påny og sorterer den i forskellige lister, som løbende skrives ud i tekstfiler, for ikke at overskride en hukkommelsesbegrænsning. Efter den har sorteret alt dataen ud i tekstfiler, sådan at den er nem at gå til, indlæser den tekstfilerne i den rækkefølge den behøver dataen, for at ændre den til det ønskede format. I nogle tilfælde laver saxparseren endnu en frasortering, baseret på data den har behandlet tidligere i processen, f.eks ved at ignorere noder der ikke er i nogle veje, eller ved at ignorere veje der ikke har nogle vejtype tilknyttet eller som ikke er en del af en kyststrækning. vores parser indeholder 1 hjælpemetode, der i sig selv har 2 hjælpemetoder, som gruppen ikke har skrevet, men fundet på \url{http://www.geodatasource.com/developers/java}, og lettere omskrevet, til vores behov. Denne metode bruges til at udregne afstanden mellem to UTM-koordinator. Derudover er vores program afhængig af klassen her:\url{https://github.com/Jotschi/geoconvert/blob/master/src/main/java/de/jotschi/geoconvert/GeoConvert.java}, som også er minimalt omskrevet. Denne bruges til at fjerne den forvridning der sker, når man prøver at få koordinator på en globe ud på et firkantet flat format. Den benytter sig af Mercator projektion, og tager UTM-koordinator som input.
