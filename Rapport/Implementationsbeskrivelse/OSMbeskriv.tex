Vores saxParser  består af følgende klasser: Saxpurger, som er hovedfilen, man skal køre, saxEventhandler som håndterer langt de fleste dele af databehandlingen, saxFilter som filtrerer noget data fra og GeoConvert, der bidrager med hjælpefunktioner mod kortforvridning. SaxParseren løber igennem forskellige stadier, Først benytter den saxFilter, der frasorterer alle mærkater pånær nogle få specificerede, der indeholder den data vil skal bruge. Det er nødvendigt at lave et tilvalgssortering frem for en frasortering, da der er så store datamængder. Herefter henter den dataen ind påny og sorterer den i forskellige lister, som løbende skrives ud i tekstfiler, for ikke at overskride en hukkommelsesbegrænsning. Efter den har sorteret alt dataen ud i tekstfiler, sådan at den er nem at gå til, indlæser den tekstfilerne i den rækkefølge den behøver dataen, for at ændre den til det ønskede format. I nogle tilfælde laver saxparseren endnu en frasortering, baseret på data den har behandlet tidligere i processen, f.eks ved at ignorere noder der ikke er i nogle veje, eller ved at ignorere veje der ikke har nogle vejtype tilknyttet eller som ikke er en del af en kyststrækning. vores parser indeholder 1 hjælpemetode, der i sig selv har 2 hjælpemetoder, som gruppen ikke har skrevet, men fundet på \url{http://www.geodatasource.com/developers/java}, og lettere omskrevet, til vores behov. Denne metode bruges til at udregne afstanden mellem to UTM-koordinator. Derudover er vores program afhængig af klassen her:\url{https://github.com/Jotschi/geoconvert/blob/master/src/main/java/de/jotschi/geoconvert/GeoConvert.java}, som også er minimalt omskrevet. Denne bruges til at fjerne den forvridning der sker, når man prøver at få koordinator på en globe ud på et firkantet flat format. Den benytter sig af Mercator projektion, og tager UTM-koordinator som input.