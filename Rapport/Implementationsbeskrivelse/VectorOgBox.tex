\section{Vector}

Klassen \emph{Vector} er en datastruktur, der tager og gemmer punkter i et x og y felt som udgør vektorens størrelse og retning. Klassen indeholder en række metoder der bl.a. tillader beregninger med vektorer.

Klassen gør det muligt at udføre aritmetiske operationer som addition, substraktion, multipikation og division. Derudover beregner metoden \emph{dist(Vector v)} afstanden mellem to vektorer, og metoden \emph{abs()} konverterer en vektors værdier til absolute værdier.

\emph{Vector} klassen har desuden også nogle hjælpemetoder der udfører handlinger ved at tage et eller flere \emph{Box} objekter.

\begin{itemize}
	\item \textbf(\emph{translate(Box from, Box to)}): Translate tager to bokse og skalerer vektoren fra den ene boks, som den befinder sig i, til den anden boks.
	\item \textbf(\emph{mirrorY(Box box)}): mirrorY spejler vektoren i forhold til y-aksen i boksen.
	\item \textbf(\emph{isInside(Box box)}): isInside tager en boks som parameter og tjekker om vektoren befinder sig i boksen.
	\item \textbf(\emph{toArray()}): toArray returnere værdien af x og y for en vektor i en tabel (array).
\end{itemize}

\section{Box}

Klassen \emph{Box} er ligeledes en datastruktur. Ved initiliasering tages og gemmes to \emph(Vector) objekter som start og stop. De to vektorer udgør øverste venstre hjørne og nedeste højre hjørne i en boks.

Metoden \emph{dimensions()} returnerer boksens dimensioner. I metoden \emph{translate(Box from, Box to)} skaleres boksen fra den boks den befinder sig i (det første parameter) til en anden boks (det andet parameter). Dette sker ved at dividerer dimensionerne fra "Box from" med dimensionerne fra "Box to" og dernæst gange dét tal med start og stop vektorerne.

\emph{flipX()} flipper x aksen og \emph{flipY()} flipper y aksen. Akserne flippes ved at bytte x eller y koordinater på start og stop vektoren med hinanden. Metoden \emph{properCorners()} bruger de to metoder til at sørge for at start vektor er det øverste venstre hjørne og stop vektor er i det nederste højre hjørne.

Metoden \emph{relativeToAbsolute(Vector relative)} konverterer vektorens relative koordinater til absolute koordinater i forhold til boksen. Omvendt konverterer metoden \emph{absoluteToRelative(Vector relative)} absolute koordinater til relative koordinater.

\emph{getCenter()} metoden finder centrum af en boks. Metoden anvendes blandt andet i metoden \emph{scale(double scalar)}, som skalerer boksen, samtidig med at centrum fastholdes.

Derudover indeholder klassen en række andre metoder:

\begin{itemize}
	\item \textbf(\emph{overlapping(Box box)}): overlapping fortæller om boksen overlappes af en anden boks.
	\item \textbf(\emph{ratio()}): boksens ratio returneres.
	\item \textbf(\emph{toArray()}): koodinaterne returneres i et to-dimensionelt array,
	\item \textbf(\emph{grow(double amount)}): grow får boksen til at vokse i alle retninger.
\end{itemize}