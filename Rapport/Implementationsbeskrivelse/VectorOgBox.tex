\section{Vector}

Klassen \emph{Vector} er en datastruktur, der tager og gemmer punkter i et x og y felt som vektorer.  Klassen indeholder en række metoder der bl.a. tillader beregninger med vektorer. 

Klassen gøre det muligt at udføre aritmetiske operationer som addition og substraktion blandt vektorer, multipikation og division blandt vektorer samt multiplikation og division mellem en vektor og skalar. Derudover beregner metoden \emph{dist(Vector v)} afstanden mellem to vektorer og metoden \emph{abs()} konverterer en vektors værdier til absolute værdier. 

\emph{Vector} klassen har desuden også nogle hjælpemetoder der udføre handlinger ved at tage et eller flere \emph{Box} objekter. 

Metoden \emph{translate(Box from, Box to)} tager to bokse og oversætter vektoren fra den ene boks, som den befinder sig i, til den anden boks. Mens en anden metode \emph{mirrorY(Box box)} spejler vektoren med y-aksen i forhold til en boks. 

\emph{isInside(Box box)} metoden tager en boks som parameter og tjekker om vektoren befinder sig i boksen.  

Metoden (\emph{toArray()}) returnere værdien af x og y for en vektor i en tabel (array).


\section{Box}

Klassen \emph{Box} er ligeledes en datastruktur. Ved initiliasering tages og gemmes to \emph(Vector) objekter som start og stop. De to vektorer udgør en boks.

Metoden \emph{dimensions()} returnerer boksens dimensioner. I metoden \emph{translate(Box from, Box to)} oversættes den nuværende boks til en anden boks. Dette sker ved at dividerer dimensionerne fra "Box to" med dimensionerne fra "Box from" og dernæst gange dét tal med start og stop vektorerne.

\emph{flipX()} flipper x aksen og \emph{flipY()} flipper y aksen. Metoden \emph{properCorners()} bruger de to metoder til at sørge for at start vektor er det øverste venstre hjørne og stop vektor er i det nederste højre hjørne.

Metoden \emph{relativeToAbsolute(Vector relative)} konverterer vektorens relative koordinater til absolute koordinater. Omvendt konverterer metoden \emph{absoluteToRelative(Vector relative)} absolute koordinater til relative koordinater.

\emph{getCenter()} metoden finder centrum af en boks. Metoden indgår bl.a. i metoden \emph{scale(double scalar)}, som skalerer boksen, samtidig med at fastholde centrum. 

Derudover indeholder klassen en række andre metoder. \emph{overlapping(Box box)} fortæller om boksen overlappes af en anden boks.\emph{ratio()} returnerer boksens ratio.  \emph{toArray()} returnerer koodinaterne i et to-dimensionelt array, \emph{grow(double amount)} får boksen til at vokse i alle retninger m.m.