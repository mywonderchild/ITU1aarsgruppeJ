\section{Vector}

Klassen \emph{Vector} er en datastruktur, der tager og gemmer punkter i et x og y felt som vektorer. Klassen indeholder en række metoder der bl.a. tillader beregninger med vektorer.
% x og y felterne er ikke vektorer men udgør tilsammen vektorens størrelse og retning.

Klassen gør det muligt at udføre aritmetiske operationer som addition, substraktion, multipikation og division blandt vektorer, samt multiplikation og division mellem en vektor og skalar. Derudover beregner metoden \emph{dist(Vector v)} afstanden mellem to vektorer, og metoden \emph{abs()} konverterer en vektors værdier til absolute værdier.
% Behøver ikke nødvendigvis at nævne alle metoder; måske skære lidt ned og beskrive mere overfladisk?

\emph{Vector} klassen har desuden også nogle hjælpemetoder der udfører handlinger ved at tage et eller flere \emph{Box} objekter.

Metoden \emph{translate(Box from, Box to)} tager to bokse og skalerer vektoren fra den ene boks, som den befinder sig i, til den anden boks. Mens en anden metode \emph{mirrorY(Box box)} spejler vektoren i forhold til y-aksen i boksen. \emph{isInside(Box box)} metoden tager en boks som parameter og tjekker om vektoren befinder sig i boksen.

Metoden (\emph{toArray()}) returnere værdien af x og y for en vektor i en tabel (array).

% Er lidt usikker på hvorvidt det er en liste af metoder eller en sammenhængende tekst. Måske burde det stilles op i punktform?

\section{Box}

Klassen \emph{Box} er ligeledes en datastruktur. Ved initiliasering tages og gemmes to \emph(Vector) objekter som start og stop. De to vektorer udgør øverste venstre hjørne og nedeste højre hjørne i en boks.

Metoden \emph{dimensions()} returnerer boksens dimensioner. I metoden \emph{translate(Box from, Box to)} skaleres boksen fra den boks den befinder sig i (det første parameter) til en anden boks (det andet parameter). Dette sker ved at dividerer dimensionerne fra "Box to" med dimensionerne fra "Box from" og dernæst gange dét tal med start og stop vektorerne.

% Omvendt. Vektorerne ganges med den skaleringsværdi der netop er blevet determineret.

\emph{flipX()} flipper x aksen og \emph{flipY()} flipper y aksen. Metoden \emph{properCorners()} bruger de to metoder til at sørge for at start vektor er det øverste venstre hjørne og stop vektor er i det nederste højre hjørne.

% Synes det ville være godt med en lidt mere sigende beskrivelse af hvad flip metoderne egentlig gør (de bytter x eller y koordinaterne på start og stop vektoren med hinanden)

Metoden \emph{relativeToAbsolute(Vector relative)} konverterer vektorens relative koordinater til absolute koordinater. Omvendt konverterer metoden \emph{absoluteToRelative(Vector relative)} absolute koordinater til relative koordinater.

% Vigtigt at nævne at det er i forhold til boksen

\emph{getCenter()} metoden finder centrum af en boks. Metoden anvendes blandt andet i metoden \emph{scale(double scalar)}, som skalerer boksen, samtidig med at centrum fastholdes.

Derudover indeholder klassen en række andre metoder. \emph{overlapping(Box box)} fortæller om boksen overlappes af en anden boks. \emph{ratio()} returnerer boksens ratio. \emph{toArray()} returnerer koodinaterne i et to-dimensionelt array, \emph{grow(double amount)} får boksen til at vokse i alle retninger.

% Igen kunne man overveje om det er bedre med punktopstilling af metoder for afsnittet generelt.