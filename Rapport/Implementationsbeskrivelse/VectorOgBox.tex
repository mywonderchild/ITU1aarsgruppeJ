\section{Vector}

Klassen \emph{Vector} er en datastruktur, der repræsenterer en vektor. Klassen indeholder en række hjælpemetoder, der bl.a. tillader beregninger med vektorer.

Klassen gør det muligt at udføre aritmetiske operationer som addition, substraktion, multipikation og division. Derudover beregner metoden \emph{dist(Vector v)} afstanden mellem to vektorer, og metoden \emph{abs()} konverterer en vektors værdier til absolute værdier.

\begin{description}
	\item [dist(Vector v)]: Beregner afstanden mellem to vektorer.
	\item [abs()]: Konverterer en vektors værdier til de tilsvarende absolutte værdier.
	\item [translate(Box from, Box to)]: Tager to bokse og skalerer vektoren fra den ene boks, som den befinder sig i, til den anden boks.
	\item [mirrorY(Box box)]: Spejler vektoren i forhold til y-aksen i boksen.
	\item [isInside(Box box)]: Tager en boks som parameter, og tjekker om vektoren befinder sig i boksen.
	\item [toArray()]: Returnerer værdien af x og y for en vektor i en tabel (array).
\end{description}

Klassen gør det også muligt at udføre aritmetiske operationer som addition, subtraktion, multiplikation og division. 

\section{Box}

Klassen \emph{Box} er ligeledes en datastruktur. En box består af to vektorer start og stop, der udgør henholdsvis øverste venstre hjørne og nedeste højre hjørne af boxen.

Derudover indeholder klassen en række metoder:

\begin{description}
	\item [dimensions()]: Returnerer boksens dimensioner.
	\item [translate(Box from, Box to)]: Skalerer fra from-boksen til to-boxen.
	\item [flipX()]: Flipper x-aksen ved at bytte om på x-koordinaterne i start og stop.
	\item [flipY()]: Flipper y-aksen ved at bytte om på y-koordinaterne i start og stop.
	\item [properCorners()]: Benytter flip-funktionerne til at sikre at start og stop repræsenterer de rigtige vectorer.
	\item [getCenter()]: Finder centrum af en boks.
	\item [scale(double scalar)]: skalerer boksen i forhold til den givne scalar.
	\item [relativeToAbsolute(Vector relative)]: Konverterer vektorens relative koordinater til absolute koordinater, i forhold til boksen.
	\item [absoluteToRelative(Vector absolute)]:Konverterer vektorens absolute koordinater, i forhold til boksen, til relative koordinater.
	\item [overlapping(Box box)]: Overlapping fortæller om boksen overlappes af en anden boks.
	\item [ratio()]: Boksens ratio returneres.
	\item [toArray()]: Boksens koordinater returneres i et to-dimensionelt array.
	\item [grow(double amount)]: Får boksen til at vokse i alle retninger.
\end{description}