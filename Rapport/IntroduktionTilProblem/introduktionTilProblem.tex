Vores opgave går ud på at lave et interaktivt Danmarkskort med indbyggede zoom og scroll-funktioner. Kortet skal have alle veje og stier i det udleverede KRAK-datasæt markeret og være i stand til at vise navnet på den vej, der er nærmest på musemarkøren. Programmet skal være i stand til at vise den hurtigste rute fra 2 udvalgte punkter på kortet, og skrive en rutevejledning til dette. Programmet skal være inituiativt og være responsivt nok, til ikke at skabe gene. Ud over KRAK-datasættet skal programmet også være kompatibel med Open Street Map (OSM) datasæt. For den præcise kravsspecifikation se \ref{sec:Kravsspecifikation}.
\newline \newline
Ud over de obligatoriske krav, har vi implementeret følgende features:
\begin{itemize}
	\item \textbf{udregn hurtigst vej til alle punkter fra givent startpunkt:} Man kan finde den hurtigste vej, ved at højreklikke på sit startspunkt, og herefter vælge sin destination ved at dobbeltklikke der. Vejen findes med det samme uden forsinkelser.
	\item \textbf{Skjulbar rutevejledningsinterface:} Vi har lavet vores interface skjul-bart, så man kan se mest muligt af kortet når dette er ønskværdigt.
	\item \textbf{dynamisk vejbredde:} Vejene angives med en bredde, der ændres alt efter zoom-niveau, så man tydeligt kan se hovedveje på kortet.
	\item \textbf{navneapproksimation:} Når man prøver at finde en rute, ved at indtaste en adresse, vil programmet foreslå adresser der minder om det skrevne, i tilfælde af tastefejl.
	\item \textbf{søgning imens man skriver:} Når man prøver at finde en rute, ved at indtaste en adresse, vil programmet løbende foreslå adresser der inhdeholder det skrevne.
	\item \textbf{Kystlinje:} Vi har tegnet Danmarks kystlinje.
\end{itemize}