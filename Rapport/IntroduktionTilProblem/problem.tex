Vores opgave går ud på at lave et interaktivt Danmarkskort, med indbyggede zoom og scroll-funktioner. Kortet skal have alle veje i det udleverede datasæt og stier markeret og være i stand til at vise den nærmeste vej, på der hvor ens cursor er. For den præcies kravsspecifikation se "LAV REFERENCE TIL BILAGET MED KRAVSSPECIFIKATIONEN"
\newline \newline
Ud over de obligatoriske krav, har vi implementeret følgende features:
\begin{itemize}
	\item{Zoom out: Man kan zoome ud på keyboardet v.h.a. o,+ eller num(+), eller ved at rulle sit mussehjul baglæns}
	\item{Zoom in: Man kan zoome ind på keyboardet v.h.a. i,- eller num(-), eller ved at rulle sit mussehjul fremad}
	\item{Scrolling: Man kan scrolle kortet på keyboardet via piletasterne eller med musen ved at venstreklikke og hive kortet rundt}
	\item{Map reset: Man kan vende tilbage til startsposition og zoomgrad på kortet v.h.a. r, backspace eller space, eller ved at trykke på midterste musseknap}
	\item{Vi har tegnet Danmarks kystlinje}
\end{itemize}