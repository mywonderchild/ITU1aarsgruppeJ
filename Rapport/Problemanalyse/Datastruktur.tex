\section{Datastruktur}
\label{sec:datastruktur}
I implementeringen af ethvert kort må den bagvedliggende datastruktur nøje overvejes, da denne i høj grad afgør applikationens ressourcekrav. Overvejelserne omkring datastrukturen involverede blandt andet muligheden for kun at indlæse specifikke dele af dataet samt søgeegenskaber. Det givne datasæt består af punkter og tilhørende linjesegmenter, som forbinder to punkter. Punkterne indeholder information om deres individuelle placering, mens linjesegmenterne er forbundet med en række forskellige oplysninger såsom vejnavn og -type. Da kortdataet består af linjesegmenter, er det også af betydning, at disse kan repræsenteres gennemførligt. Værd at overveje var også datastørrelsen - datakilden fra Krak indeholder over en million punkter, og disse skal på ethvert tidspunkt under programkørslen kunne tilgås hurtigt.

To forskellige datastrukturer syntes besidde de nævnte egenskaber: en sorteret tabel og et quadtree. Den sorterede tabel brillere med en ukompliceret implementering. En quadtree-struktur, mens kompliceret at implementere, udmærker sig derimod ved dens simple datasegmentering samt hastige indlæsning af specifikke områder. Derudover er det muligt at justere centrale dele af en quadtree-struktur, så denne indpasses datakilden. Det vurderedes at quadtree-strukturen ville udmunde i en både hurtigere og mere dynamisk datastruktur, og således sås som den mest passende løsning.
% Forklar mere specifikt, hvad et quadtree er og hvordan det virker?