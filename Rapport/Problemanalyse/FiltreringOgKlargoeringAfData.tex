\section{Filtrering og klargøring af data}
\label{sec:filtreringOgKlargoeringAfData}

Vi udviklede to moduler til at filtrere og klargøre data --- et for hvert datasæt. Fælles funktionalitet for de to moduler er at:

\begin{itemize}
	\item Frasortere unødvendige data.
	\item Ensrette formatet som data gemmes i.
	\item Konvertere koordinater således at de ligger inden for et 1000 gange 1000 koordinatsystem med udgangspunkt i øverste venstre hjørne.
	\item Maks x og y værdier gemmes til brug ved indlæsning.
	\item Rette op på evt. fejl og mangler i datasæt.
\end{itemize}

\subsection{Krak}
\label{sec:krak}

For nogle af vejstykkerne i datasættet var hastigheden 0 km/t, hvilket vi anser som en fejl, i de data vi har modtaget. Dette viste sig at være problematisk, eftersom vores rutealgoritme bruger vejstykkernes hastigheder til at beregne den hurtigste vej.

For at afhjælpe dette problem valgte vi at beregne hastigheden ud fra vejstykkets længde og estimerede køretid med følgende formel:

\vspace{1ex}
$s = \frac{\frac{l}{1000}}{\frac{t}{60} \cdot \frac{1}{1.15}}$
\vspace{1ex}

Hvor $s$ er hastigheden i km/t, $l$ er længden i meter og $t$ er den estimerede køretid i minutter. Formlen tager højde for at Krak har lagt 15\% til deres estimerede køretider. Denne løsning er ikke perfekt, da det havde været optimalt at aflæse alle hastigheder frem for at udlede dem, men var et nødvendigt onde for at opnå den ønskede funktionalitet.

De kystlinjedata vi fik fra en anden gruppe var angivet som en serie af forbindelser mellem koordinater: $(x_1, y_1), (x_2, y_2)$. Disse data bliver i modulet splittet ud i en række punkter og vejstykker.

\subsection{Open Street Maps}
\label{sec:openStreetMaps}
Da vi havde designet vores første del af programmet ud fra KRAK-datasættet, valgte vi at benytte dettes datastruktur som det ønskede. Derfor havde vi et behov for at omforme OSM-datasættet til det ønskede format. Der var dog flere forhindringer ved dette. For det første var mængden af data i OSM-datasættet enormt. Der var så meget data at ingen af vores pc'ere var i stand til at kunne læse hele datasættet på en gang. Derudover var der  små differencer på datasættet, og hvad der var beskrevet i dokumentationen. F.eks rækkefølgen af forskellige værdier, i en node. Til sidst var datastrukturen anderledes bygget op end KRAK, så der var andre måder at repræsentere f.eks en vej på. Alle disse hindringer skabte en følelse af at information om dataen var utilgængelig, da den først kunne ses, efter man havde sorteret i den, og det var svært at vide hvad man skulle sortere efter.

Til sortering af Data benyttede vi os af en SAX-parser, der sorterede i XML-filen, og skrev den ønskede data ned. I starten var denne parser optimeret mod hastighed, ved at gemme den ønskede data i Arraylists før den blev skrevet til en outputfil, men hukkommelsesforbruget blev hurtigt for stort, og der blev til sidst implementeret en noget langsommmere løsning, hvor parseren skriver og læser løbende fra filer, sådan at hukkommelsen aldrig bliver fyldt, fordi at man bruger fastpladelageret som en udvidelse til hukommelsen. Generelt lykkedes det os at få store dele af dataen vi ønskede med over, men der var dog nogle ting vi ikke kunne få i fyldestgørende grad:

\begin{itemize}
	\item \textbf{Vejhastigheder:} Det mest sigende tag vi kunne finde, var knyttet til vej-objekter i OSM. Desværre var det kun 1/3 - 1/2 af vejene der havde dette tag markeret. Resten havde ingen hastighedsmarkering. Vi fandt i dokumententationen et tegn på at man kunne gå ud fra visse hastigheder i byzoner og lignende. Desværre kunne vi ikke finde belæg for at sådanne zoner var tydeligt markeret i datasættet. Da vi har punkter der markerer byer, ville vi kunne lave en antagelse, der hedder at alle vejdele der er inden for x afstand fra dette punkt er inde for bygrænsen, men denne antagelse ville også bare være løse gæt, og derved heller ikke give et sandfærdigt svar.
	\item \textbf{postnumre knyttet til veje:} Vi skulle benytte denne information i tilfælde at flere veje havde samme navn. Desværre understøttede OSM ikke dette, da OSM, havde knyttet deres postadresser til adressepunkter, der ikke var forbundet med veje. Igen ledte vi efter en potentiel zonemarkering, men kunne ikke finde belæg for at sådanne zoner var tydeligt markeret i datasættet. Man ville kunne prøve at finde nærmeste adressepunkt, på en vejdel, og så give det postnummer, som denne havde, men denne funktionalitet, har vi ikke implementeret.
	\item \textbf{mapstretch:} Grundet jordens grumning, vil et datasæt baseret på længde og breddegrader bliver forstrukket, når det bliver præsenteret på et firkantet flat kort. Man skal her vælge om man vil bibeholde landmassernes egentlige størrelse, eller vil bibeholde at graderne forbliver paralelle linjer, med ens spredning imellem punkterne. Her kunne man f.eks nytte Mercator-projektion, for at ændre billedet til det ønskede format.
\end{itemize}