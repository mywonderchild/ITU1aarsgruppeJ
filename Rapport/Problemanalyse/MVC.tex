Vi har valgt at designe vores program ud fra tanken omkring Model-View-Controllerdesignprincippet. Dette sikrer en stærk adskillelse mellem funktionalitet, input og repræsentation, hvilket hjælper til et mere modulært miljø, der giver mere testbar kode, Forståelse for en klasses opgaver, såvel som brugbarhed i senere projekter. Vi endte med at benytte supervising Controller, da det var ønskværdigt at gøre view så modulær, og strømlinet som muligt.\\
\\
Af andre designparadigmer, der også kunne være benyttet, kan nævnes Model View ViewModel, der adskiller de funktioner vi har lagt i vores controller ud i henholdsvist ViewModel og Binder, og Presentration-Abstraction-Control, der benytter sig af et mere menneskenært designparadigme, hvor man designer efter menneskelige abstraktioner. Dette paradigme var dog ikke nærliggende for dette projekt, da vi kun beskæftiger os med 1 abstraktion, nemlig vejkortet. Derfor ville vores færdige resultat minde om det nuværende, med kun en agent.