Da vi havde designet vores første del af programmet ud fra KRAK-datasættet, valgte vi at benytte dettes datastruktur som udgangspunktet for vores datastruktur. Derfor havde vi et behov for at omforme OSM-datasættet til dette format. Der var dog flere forhindringer ved dette. For det første var mængden af data i OSM-datasættet enormt. Der var så meget data at ingen af vores pc'ere var i stand til at kunne læse hele datasættet på en gang. Derudover var der  små differencer på datasættet, og hvad der var beskrevet i dokumentationen. F.eks rækkefølgen af forskellige værdier, i en node. Til sidst var datastrukturen anderledes bygget op end KRAK, så der var andre måder at repræsentere f.eks en vej på. Alle disse hindringer skabte en følelse af at information om dataen var utilgængelig, da den først kunne ses, efter man havde sorteret i den, og det var svært at vide hvad man skulle sortere efter.

Til sortering af Data benyttede vi os af en SAX-parser, der sorterede i XML-filen, og skrev den ønskede data ned. I starten var denne parser optimeret mod hastighed, ved at gemme den ønskede data i Arraylists før den blev skrevet til en outputfil, men hukkommelsesforbruget blev hurtigt for stort, og der blev til sidst implementeret en noget langsommmere løsning, hvor parseren skriver og læser løbende fra filer, sådan at hukkommelsen aldrig bliver fyldt, fordi at man bruger fastpladelageret som en udvidelse til hukommelsen. Generelt lykkedes det os at få store dele af dataen vi ønskede med over, men der var dog nogle ting vi ikke kunne få til at virke i fyldestgørende grad:

\begin{itemize}
	\item \textbf{Vejhastigheder:} Det mest sigende mærkat(tag) vi kunne finde, var knyttet til vej-objekter(way) i OSM. Desværre var det kun 1/3 - 1/2 af vejene der havde dette mærkat markeret. Resten havde ingen hastighedsmarkering. Vi fandt i dokumententationen et tegn på at man kunne gå ud fra visse hastigheder i byzoner og lignende. Desværre kunne vi ikke finde belæg for at sådanne zoner var tydeligt markeret i datasættet. Da vi har punkter der markerer byer, ville vi kunne lave en antagelse, der hedder at alle vejdele(edges) der er inden for x afstand fra dette punkt er inde for bygrænsen, men denne antagelse ville også bare være løse gæt, og derved heller ikke give et sandfærdigt indtryk.
	\item \textbf{postnumre knyttet til veje:} Vi skulle benytte denne information i tilfælde at flere veje havde samme navn. Desværre understøttede OSM ikke dette, da OSM, havde knyttet deres postadresser til adressepunkter, der ikke var forbundet med veje. Igen ledte vi efter en potentiel zonemarkering, men kunne ikke finde belæg for at sådanne zoner var tydeligt markeret i datasættet. Man ville kunne prøve at finde nærmeste adressepunkt, på en vejdel, og så give det postnummer, som denne havde, men denne funktionalitet, har vi ikke implementeret.
\end{itemize}

Vi har valgt begrænse os til de noder(node), der er en del af en vej, og de veje der indeholder et vejtypemærkat. Metadata i OSM-datasættet var generelt gemt i en række mærkater, der består af en nøgle og en værdi. antallet af mærkatnøgler er dog utrolig stort, og der er ikke nogle form for praktisk begrænsning af brugen af nøglerne. Vi har efter længere research valgt at gøre brug af følgende nøgler:

\begin{itemize}
	\item \textbf{name:} Dette mærkat indeholder vejnavnet
	\item \textbf{highway:} Dette mærkat indeholder vejtypen(motorvej, landevej, gågade etc.)
	\item \textbf{maxspeed:} Dette mærkat indeholder den højeste tilladte transporthastighed
	\item \textbf{natural=coast:} Dette mærkat indeholder kystlinjen
\end{itemize}
For at få et scope 0->1000 på vores xy-koordinatsæt, måtte vi benytte følgende formel på vores UTM-koordinater $newCoordinate = \\(UTMCoordinate-lowestUTMCoordinate) * (1000/(highestUTMCoordinate-lowestUTMCoordinate)) $\\ Derudover spejler vi længdebredderne, så verdenshjørnerne passer til den almindelige kortstandard.
