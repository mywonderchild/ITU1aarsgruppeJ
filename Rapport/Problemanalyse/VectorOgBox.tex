\section{Vector og Box}

Indledningsvist valgte vi at repræsentere vektorer og bokse (rektangler) som henholdsvis én- og flerdimensionale tabeller (arrays).
	
Dette virkede umiddelbart som et smart valg, da man på den måde kunne nøjes med at have én variabel per punkt eller boks. Det viste sig dog at være problematisk, efterhånden som vi fik skrevet kode, der foretog beregninger med vektorer og bokse. Dette skyldtes at de indeks, som værdierne havde i tabellen, ikke fortalte noget, om hvad værdien indeholdt.

Et andet problem var, at vi foretog de samme operationer på tabellerne mange gange i forskellige klasser. Vi indså derfor, at vi havde brug for at samle funktionaliteten, der havde at gøre med vektorer og bokse. Herved undgik vi unødig kode-duplikering, der gjorde kodebasen svær at vedligeholde.

Vores løsning på begge ovenstående problemstillinger blev at implementere to klasser, \emph{Vector} og \emph{Box}, der begge er datastrukturer med tilhørende hjælpemetoder.

Ved at implementere datastrukturer fik vi løst problemet med navngivningen af værdier, der nu blev mere sigende. For eksempel gik vektorens koordinatsæt fra at være værdien tilhørende indeks 0 og 1 i en tabel, til at være værdien tilknyttet x og y feltet i et \emph{Vector} objekt.

Ved at implementere hjælpefunktioner fik vi løst problemet med kode-duplikering. Derudover muliggjorde det afprøvning af funktionaliteten, hvilket viste sig at være meget brugbart, da der i perioder under udviklingen var mange fejl i den kode, der var afhængig af vektorer og bokse, som var svære at udbedre. Ved at afprøve \emph{Vector} og \emph{Box} klassernes hjælpefunktioner havde vi et solidt udgangspunkt og kunne koncentrere os om de logiske fejl.

Endelig gjorde det arbejdet med datatyperne nemmere, at vi designede hjælpemetoderne, således at man kunne sammenkæde metoder.