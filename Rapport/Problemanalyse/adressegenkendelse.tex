\section{Adressegenkendelse}
For ikke blot at kunne definere start- og slutpunktet for rutenavigation med markøren, var det nødvendigt at implementere manuel adresseindtastning. Implementeringen byggede indledningsvist på opretholdelsen af et hashmap med adressestrenge og dertilhørende vejsegmenter. Således kunne en eftersøgt adresse hurtigt forbindes til en række linjer. Præcisionen af brugerens indtastning var altafgørende for løsningens effektivitet, da det indtastede skulle stemme fuldstændig overens med navnet på den søgte adresse. Derfor fandt vi det nyttigt at implementere adressegenkendelse, så korrektheden af den indtastede adresse ikke begrænsede brugeren. Til formålet skrev vi en ordinær algoritme for \emph{Levenshtein-distance}, som sammenligner to ord bogstav for bogstav og returnere distancen mellem disse. Distancen mellem to ord forøges med én hver gang et bogstav slettes eller indsættes og med to hver gang et bogstav substitueres. Udover denne normale distancebetragtning, ønskede vi at muliggøre omkostningsfri autofuldførelse af adresser. Derfor forøger indsættelse ikke distancen såfremt de foregående bogstaver var ens mellem ordene --- altså når distancen i forvejen er nul. Resultatet af dette er, at distancen mellem ''rued lang'' og ''rued langgaards vej'' er nul.

Da antallet af unikke adresser er ganske højt, var det køretidsmæssigt tungt at sammenligne en indtastning med samtlige kendte adresser. Dette var ganske mærkbart under brugerindtastning, da det forsagede en kort forsinkelse efter hver karakterindtastning. Vi løste indledningsvist dette ved først at påbegynde adressegenkendelsen, når brugeren ikke havde foretaget en indtastning i et kort tidsrum. Senere optimerede vi denne løsning ved simpelthen at placere kørslen af adressegenkendelse på en separat tråd, som ikke fastlåste brugergrænsefladen under kørsel. Herved kunne en ny adressegenkendelse påbegyndes (og erstatte den gamle), hver gang brugeren foretog en ændring i indtastningen. Dette førte til en mere flydende indtastning samt minimal ventetid på adresseforslag.