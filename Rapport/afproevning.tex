\documentclass{article}

\usepackage[utf8]{inputenc}

\begin{document}
	\section{Afprøvning}
	\label{sec:afproevning}
	I bestræbelse efter sikring af applikationens kvalitet og pålidelighed er en række afprøvninger af dennes bestanddele udført løbende med udviklingen. Afprøvningen er ikke foretaget med en målsætning om at udtømme samtlige muligheder for systemfejl, men derimod blot for fyldestgørende at forhindre invaliderende defekter samt bidrage til udviklingsprocessen.

	\subsection{Metode}
	\label{subsec:metode}
	% evt. nævn, at der findes et hav af testmetoder
	Da afprøvningsmetoden, som anvendes under udarbejdelsen af programprøver, er afgørende for afprøvningens dækningsgrad, fastsattes den indledningsvist som \emph{white-box testing} med \emph{branch coverage}. Det er således påkrævet at samtlige kodesegmenter i en afprøvet kodeblok køres én gang som minimum.
	
	Da udviklingen er foretaget i Java, er afprøvning udført i samspil med prøvemiljøet JUnit - et populært Java værktøj til netop afprøvning. JUnit er således anvendt under kørsel af prøverne, hvorfor disse også er skrevet op imod JUnits API.

	\subsection{Ækvivalensklasser}
	\label{subsec:aekvivalensklasser}
\end{document}